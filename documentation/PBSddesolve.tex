\documentclass[letterpaper]{book}
\usepackage[times,inconsolata,hyper]{Rd}
\usepackage{makeidx}
\usepackage[utf8,latin1]{inputenc}
% \usepackage{graphicx} % @USE GRAPHICX@
\makeindex{}
\topmargin -0.25in \oddsidemargin 0in \evensidemargin 0in
\textheight 9in \textwidth 6.5in
\begin{document}
\setcounter{page}{13}
\chapter*{}
\begin{center}
{\textbf{\huge \R{} documentation}} \par\bigskip{{\Large of all in \file{PBSddesolve}}}
\par\bigskip{\large \today}
\end{center}
\Rdcontents{\R{} topics documented:}
\inputencoding{utf8}
\HeaderA{dde}{Solve Delay Differential Equations}{dde}
\keyword{math}{dde}
%
\begin{Description}\relax
A solver for systems of delay differential equations based on
numerical routines from Simon Wood's \emph{solv95} program. This
solver is also capable of solving systems of ordinary differential
equations. 

Please see the included demos for examples of how to use \code{dde}.

To view available demos run \code{demo(package="PBSddesolve")}.\\{}
The supplied demos require that the R package \pkg{PBSmodelling}
be installed.
\end{Description}
%
\begin{Usage}
\begin{verbatim}
dde(y, times, func, parms=NULL, switchfunc=NULL, mapfunc=NULL, 
    tol=1e-08, dt=0.1, hbsize=10000) 
\end{verbatim}
\end{Usage}
%
\begin{Arguments}
\begin{ldescription}
\item[\code{y}] Vector of initial values of the DDE system. The size of the
supplied vector determines the number of variables in the system.
\item[\code{times}] Numeric vector of specific times to solve. 
\item[\code{func}] A user supplied function that computes the gradients in
the DDE system at time \code{t}. The function must be defined
using the arguments: \code{(t,y)} or \code{(t,y,parms)}, where
\code{t} is the current time in the integration, \code{y} is a
vector of the current estimated variables of the DDE system, and
\code{parms} is any R object representing additional parameters
(optional).

The argument \code{func} must return one of the two following
return types: \\{}
1) a vector containing the calculated gradients for each variable;
or \\{}
2) a list with two elements - the first a vector of calculated
gradients, the second a vector (possibly named) of values for a
variable specified by the user at each point in the integration.
\item[\code{parms}] Any constant parameters to pass to \code{func},
\code{switchfunc}, and \code{mapfunc}.
\item[\code{switchfunc}] An optional function that is used to manipulate
state values at given times. The switch function takes the
arguments \code{(t,y)} or \code{(t,y,parms)} and must return a
numeric vector. The size of the vector determines the number of
switches used by the model. As values of \code{switchfunc} pass
through zero (from positive to negative), a corresponding call to
\code{mapfunc} is made, which can then modify any state value.
\item[\code{mapfunc}] If \code{switchfunc} is defined, then a map function
must also be supplied with arguments \code{(t, y, switch\_id)} or
\code{t, y, switch\_id, parms)}, where \code{t} is the time,
\code{y} are the current state values, \code{switch\_id} is the
index of the triggered switch, and \code{parms} are additional
constant parameters.
\item[\code{tol}] Maximum error tolerated at each time step (as a
proportion of the state variable concerned).
\item[\code{dt}] Maximum initial time step.
\item[\code{hbsize}] History buffer size required for solving DDEs.
\end{ldescription}
\end{Arguments}
%
\begin{Details}\relax
The user supplied function \code{func} can access past values (lags)
of \code{y} by calling the \code{\LinkA{pastvalue}{pastvalue}}
function. Past gradients are accessible by the 
\code{\LinkA{pastgradient}{pastgradient}} function. These functions 
can only be called from \code{func} and can only be passed values
of \code{t} greater or equal to the start time, but less than the
current time of the integration point. For example, calling 
\code{pastvalue(t)} is not allowed, since these values are the
current values which are passed in as \code{y}.
\end{Details}
%
\begin{Value}
A data frame with one column for \code{t}, a column for every
variable in the system, and a column for every additional value that
may (or may not) have been returned by \code{func} in the second
element of the list.

If the initial \code{y} values parameter was named, then the solved
values column will use the same names. Otherwise \code{y1},
\code{y2}, ... will be used.

If \code{func} returned a list, with a named vector as the second
element, then those names will be used as the column names. If the
vector was not named, then \code{extra1}, \code{extra2}, ... will be
used.
\end{Value}
%
\begin{SeeAlso}\relax
\code{\LinkA{pastvalue}{pastvalue}}
\end{SeeAlso}
%
\begin{Examples}
\begin{ExampleCode}
##################################################
# This is just a single example of using dde.
# For more examples see demo(package="PBSddesolve")
# the demos require the package PBSmodelling
##################################################

require(PBSddesolve)
local(env=.PBSddeEnv, expr={
  #create a func to return dde gradient
  yprime <- function(t,y,parms) {
    if (t < parms$tau)
      lag <- parms$initial
    else
      lag <- pastvalue(t - parms$tau)
    y1 <- parms$a * y[1] - (y[1]^3/3) + parms$m * (lag[1] - y[1])
    y2 <- y[1] - y[2]
    return(c(y1,y2))
  }

  #define initial values and parameters
  yinit <- c(1,1)
  parms <- list(tau=3, a=2, m=-10, initial=yinit)

  # solve the dde system
  yout <- dde(y=yinit,times=seq(0,30,0.1),func=yprime,parms=parms)

  # and display the results
  plot(yout$time, yout$y1, type="l", col="red", xlab="t", ylab="y", 
    ylim=c(min(yout$y1, yout$y2), max(yout$y1, yout$y2)))
  lines(yout$time, yout$y2, col="blue")
  legend("topleft", legend = c("y1", "y2"),lwd=2, lty = 1, 
    xjust = 1, yjust = 1, col = c("red","blue"))
})
\end{ExampleCode}
\end{Examples}
\inputencoding{utf8}
\HeaderA{pastvalue}{Retrieve Past Values (lags) During Gradient Calculation}{pastvalue}
\aliasA{pastgradient}{pastvalue}{pastgradient}
\keyword{math}{pastvalue}
%
\begin{Description}\relax
These routines provides access to variable history at lagged times.
The lagged time \eqn{t}{} must not be less than \eqn{t_0}{}, nor
should it be greater than the current time of gradient calculation.
The routine cannot be directly called by a user, and will only work
during the integration process as triggered by the \code{dde} routine.
\end{Description}
%
\begin{Usage}
\begin{verbatim}
pastvalue(t)
pastgradient(t)
\end{verbatim}
\end{Usage}
%
\begin{Arguments}
\begin{ldescription}
\item[\code{t}] Access history at time \code{t}.
\end{ldescription}
\end{Arguments}
%
\begin{Value}
Vector of variable history at time \code{t}.
\end{Value}
%
\begin{SeeAlso}\relax
\code{\LinkA{dde}{dde}}
\end{SeeAlso}
\inputencoding{utf8}
\HeaderA{PBSddesolve}{Package: Solver for Delay Differential Equations}{PBSddesolve}
\aliasA{PBSddesolve-package}{PBSddesolve}{PBSddesolve.Rdash.package}
\keyword{package}{PBSddesolve}
%
\begin{Description}\relax
A solver for systems of delay differential equations based on 
numerical routines from Simon Wood's \code{solv95} program. 
This solver is also capable of solving systems of ordinary
differential equations.
\end{Description}
%
\begin{Details}\relax
Please see the user guide \code{PBSddesolve-UG.pdf}, located in R's
library directory \code{./library/PBSddesolve/doc}, for a 
comprehensive overview.
\end{Details}
%
\begin{Author}\relax
Alex Couture-Beil <alex@mofo.ca> \\{}
Jon T. Schnute <schnutej-dfo@shaw.ca> \\{}
Rowan Haigh <rowan.haigh@dfo-mpo.gc.ca>

Maintainer: Rowan Haigh <rowan.haigh@dfo-mpo.gc.ca>
\end{Author}
%
\begin{References}\relax
Wood, S.N. (1999) Solv95: a numerical solver for systems of delay 
differential equations with switches. Saint Andrews, UK. 10 pp.

\end{References}
%
\begin{SeeAlso}\relax
\code{\LinkA{dde}{dde}}
\end{SeeAlso}
\printindex{}
\end{document}
